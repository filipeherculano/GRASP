% 
% Modelo de artigo científico para a revista   v1.1
%
% PET - Automação e Sistemas da Escola Politécnica da Universidade de São Paulo
%
% e-mail do grupo: petmecatronica@gmail.com
% e-mail da revista: revista @gmail.com
%


\documentclass[
	% -- opções da classe memoir --
	article,			% indica que é um artigo acadêmico
	12pt,				% tamanho da fonte
	oneside,			% para impressão apenas no verso. Oposto a twoside
	a4paper,			% tamanho do papel. 
	% -- opções do pacote babel --
	english,			% idioma adicional para hifenização
	brazil,				% o último idioma é o principal do documento
	sumario=tradicional
	]{abntex2}


% ---
% PACOTES
% ---

% ---
% Pacotes fundamentais 
% ---
\usepackage{lmodern}			% Usa a fonte Latin Modern
\usepackage[T1]{fontenc}		% Selecao de codigos de fonte.
\usepackage[utf8]{inputenc}		% Codificacao do documento (conversão automática dos acentos)
\usepackage{nomencl} 			% Lista de simbolos
\usepackage{color}				% Controle das cores
\usepackage{graphicx}			% Inclusão de gráficos
\usepackage{microtype} 			% Para melhorias de justificação
\usepackage{float}				% Para ajuste na posição de
\usepackage[utf8]{inputenc}
\usepackage{authblk}
% ---
		
% ---
% Pacotes adicionais, usados apenas no âmbito do Modelo Canônico do abnteX2
% ---
\usepackage{lipsum}				% para geração de dummy text
% ---
		
% ---
% Pacotes de citações
% ---
\usepackage[alf]{abntex2cite}	% Citações padrão ABNT
% ---

% ---
% Informações de dados para CAPA e FOLHA DE ROSTO
% ---

\titulo{ 
RESOLVENDO O PROBLEMA DO CAIXEIRO VIAJANTE VIA PROCEDIMENTO DE METAHEURÍSTICA GRASP }
\autor{Gabriel Angelo Freire Gonçalves,\\Nome do segundo autor}
\local{Brasil}
\date{20 de Novembro de 2018}
\affil{Universidade Estadual do Ceará \\ Departamento de ciência da computação}
% ---

% ---
% Alterando o aspecto da cor azul
% ---
\definecolor{blue}{RGB}{41,5,195}
% ---

% ---
% Informações do PDF
% ---
\makeatletter
\hypersetup{
     	%pagebackref=true,
		pdftitle={\@title}, 
		pdfauthor={\@author},
    	pdfsubject={Modelo de artigo científico com abnTeX2},
	    pdfcreator={LaTeX with abnTeX2},
		pdfkeywords={abnt}{latex}{abntex}{abntex2}{artigo científico}, 
		colorlinks=true,       		% false: boxed links; true: colored links
    	linkcolor=blue,          	% color of internal links
    	citecolor=blue,        		% color of links to bibliography
    	filecolor=magenta,      		% color of file links
		urlcolor=blue,
		bookmarksdepth=4
}
\makeatother
% --- 

% ---
% Compila o índice
% ---
\makeindex
% ---

% ---
% Altera as margens padrões
% ---
\setlrmarginsandblock{2cm}{2cm}{*}
\setulmarginsandblock{2cm}{2cm}{*}
\checkandfixthelayout
% ---

% --- 
% Espaçamentos entre linhas e parágrafos 
% --- 

% O tamanho do parágrafo é dado por:
\setlength{\parindent}{.5cm}

% Controle do espaçamento entre um parágrafo e outro:
\setlength{\parskip}{0.2cm}  % tente também \onelineskip

% Espaçamento simples
\SingleSpacing
% ---

% --- 
% Cabeçalho 
% --- 
\makepagestyle{meuestilo}
  \makeoddhead{meuestilo} %%pagina ímpar ou com oneside
     { }
     {}
     {Artigo 0,  pág. \thepage}
% ---

% ---
% Margem para resumo, palavras-chave, abstract e keywords
% ---
\def\changemargin#1#2{\list{}{\rightmargin#2\leftmargin#1}\item[]}
\let\endchangemargin=\endlist 
% ----

% ---
% Início do documento
% ---
\begin{document}

% ----------------------------------------------------------
% ELEMENTOS TEXTUAIS
% ----------------------------------------------------------
\textual

% Aplica o cabeçalho em todas as páginas, excetuando-se a primeira
\pagestyle{meuestilo}

% Retira espaço extra obsoleto entre as frases.
\frenchspacing 

% Página de titulo
\maketitle

% Aplica cabeçalho na primeira página
\thispagestyle{meuestilo}
% ---

% -----------------------------------------------------------
% Resumo em português
% -----------------------------------------------------------
\begin{changemargin}{1cm}{1cm} 
 \textbf{Resumo} – A maioria dos problemas de otimizacão combinatória pertencem à classe NP, o que significa que asua solução ótima por meio de métodos enumerativos pode exigir um tempo de processamento inviável. Assim, o uso de heurísticas é frequentemente mais interessante, pois são capazes de encontrar uma relação custo/benefício aceitável entre a qualidade da solução e o tempo de processamento. Este artigo propõe uma meta-heurística para solução do problema do caixeiro viajante (TSP), em que um procedimento de busca adaptativa aleatória gulosa(GRASP) é implementado.

 \vspace{\onelineskip}
 
 \noindent
 \textbf{Palavras-chave} – Problema do caixeiro viajante, redes auto-organizáveis, otimização combinatória; 
\end{changemargin}

% ---

% ----------------------------------------------------------
% Introdução
% ----------------------------------------------------------
\section{Introdução}
\addcontentsline{toc}{section}{Introdução}

% ---
% Nota de rodapé na primeira página
% ---

% ---

O estudo de problemas de otimização combinatória é relevante tanto do ponto de vista teórico quanto prático. Estes problemas são comumente encontrados em diversas situações reais,tais como, roteamento de veículos (“Vehicle Routing Problems” (VRP)) e programação de tabela de horários (“Scheduling”), e são usualmente classificados como NP-completo ou NP-difícil 
 (Lewisand Papadimitriou, 1997), o que implica na não existência de uma solução determinística capaz de ser executada em tempo polinomial. Por isso, ouso de heuríssticas é frequentemente mais interessante (Glover and Kochenberger, 2003), pois são capazes de encontrar uma relação
 custo/benefício aceitável entre a qualidade da solução e o tempo necessário para o seu cálculo.

\section{Definição do problema}

O Problema do Caixeiro Viajante (TSP, do inglês Traveling-Salesman Problem) é um problema
NP-Difícil da área de otimização que possui inúmeras aplicações práticas. O TSP modela diversas
situações reais, como o roteamento de veículos para atender chamados ou ocorrências.
Por pertencer à classe dos problemas NP-Difíceis, não há algoritmos eficientes para resolvê-lo
na exatidão. O TSP também é encontrado como subproblema de modelagens maiores, que envolvem
muitas vezes vários problemas pertencentes às classes NP-Completa e NP-Difícil, como
o Problema da Mochila (Martello e Toth, 1990). 

O TSP consiste em, dado um grafo completo G(V, E), com n vértices, obter um ciclo
hamiltoniano de custo mínimo, isto é, deseja-se, a partir de um vértice inicial, passar por todos
os demais vértices do grafo uma única vez e então retornar ao vértice inicial. Cada aresta que
liga um par de vértices do grafo possui um custo c(i, j) que determina o quanto se gasta para ir de i até j. Esse custo pode ter diversos significados, de acordo com a aplicação desejada.
Ele pode representar, por exemplo, a distância, o tempo ou mesmo o preço para se deslocar
entre duas cidades i e j. Considerando esses custos, o objetivo do problema é obter um ciclo
hamiltoniano de custo total mínimo a partir de um dado ponto inicial. Esse custo total é dado
pela soma dos pesos de todas as arestas contidas no ciclo. \\
\\*
A definição de um ciclo hamiltoniano é dada da seguinte forma:
Dado um grafo G(V, E) com n vértices, deseja-se obter um ciclo (v1, . . . , vn) que parta de
um vértice inicial v1, passe por todos os demais n-1 vértices de V uma única vez e retorne a v1.
A definição formal pode ser escrita como se segue:
Dados um grafo completo G(V, E) e uma função custo c : V x V → Q+, encontrar uma
permutação (v1, v2, . . . , vn) de V tal que
\\
\tabcc
\[c(vn, v1) +
\sum_{i=1}^{n-1} c(vi, vi+1)$
seja mínimo.


\section{Descrição do algoritmo e complexidade}
        digite aqui.

\section {Resultados dos experimentos computacionais}
    digite aqui 
    
\section {Conclusões}
    digite aqui


%Para mais informações sobre como citar adequadamente diferentes tipos de trabalho, consulte por exemplo as diretrizes compiladas pela Divisão de Bibliotecas da Escola Politécnica da USP (2013), disponível em

%http://www.poli.usp.br/images/stories/media/download/bibliotecas/DiretrizesTesesDissertacoes.pdf

\section*{Referências}

\noindent Blum, and Roli, A. (2003). Metaheuristics incombinatorial optimization: Overview and conceptual comparison.

\noindent SLin, S. and Kernighan, B. (1973). An effective heuristic
algorithm for the traveling-salesman problem,
Operations Research 21(2): 498–516

\vspace{-8mm}
\begingroup
\makeatletter
\renewcommand{\chapter}{\@gobbletwo}
\makeatother
\bibliographystyle{abntex2-alf-mod}
\bibliography{mybibfile}
\endgroup

\begin{changemargin}{1cm}{1cm} 



\end{changemargin}

% \begin{figure}[h]
%     \includegraphics[width=0.5\textwidth]{figure2}
%     \caption{figura2}
%     \label{figura2}
% \end{figure}

% \begin{figure}[h]
%     \includegraphics[width=0.5\textwidth]{figure2}
%     \caption{figura2}
%     \label{figura2}
% \end{figure}


\end{document}